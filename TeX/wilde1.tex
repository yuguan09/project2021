\documentclass[xelatex,ja=standard,jafont=noto]{bxjsarticle}
\usepackage[utf8]{inputenc}

\usepackage{amsmath}
\usepackage{amsthm}
\usepackage{amssymb}
\usepackage{mathrsfs}
\usepackage{graphicx} 
\usepackage{braket}


\newtheorem{theorem}{Theorem}
\newtheorem{corollary}{Corollary}
\newtheorem{lemma}{Lemma}
\newtheorem{example}{Ex\documentclass{article}
	\usepackage{CJKutf8}
	\usepackage{amsmath}
	\usepackage{amssymb}
	ample}
\newtheorem{definition}{Definition}

\def\ds{\displaystyle}
\def\ul{\underline}
	\title{ノートまとめ}
	\author{bq18026\\関宇}
	\date{3.5,2021}
	
	
\begin{document}
\maketitle


Composite quantum systems\\

consider the composite quantum$\ket{\Phi^{+}}_{AB}:$

\begin{equation}
    \ket{\Phi^{+}}_{AB}=\frac{1}{\sqrt{2}}(\ket{0}_{A}\ket{0}_{B}+\ket{1}_{A}\ket{1}_{B})
\end{equation}

Show that the entangled state has the following representation in
the +/- basis:

\begin{equation}
    \ket{\Phi^{+}}_{AB}=\frac{1}{\sqrt{2}}(\ket{+}_{A}\ket{+}_{B}+\ket{-}_{A}\ket{-}_{B})
\end{equation}

\begin{equation}
    \begin{aligned}
    &\ket{+}=\frac{1}{\sqrt{2}}\left(\begin{array}{c} 
    1  \\ 
    1
\end{array}\right),\ket{-}=\frac{1}{\sqrt{2}}\left(\begin{array}{c} 
    1  \\ 
    -1
\end{array}\right)\\
\\
&\ket{+}\otimes\ket{+}=\left(\begin{array}{c} 
   1/2  \\ 
    1/2 \\
     1/2  \\
    1/2  
\end{array}\right),\ket{-}\otimes\ket{-}=\left(\begin{array}{c} 
   1/2  \\ 
    -1/2 \\
     -1/2  \\
    1/2  
\end{array}\right)\\
\\
&\frac{1}{\sqrt{2}}(\ket{+}_{A}\ket{+}_{B}+\ket{-}_{A}\ket{-}_{B})\\
\\
&=\frac{1}{\sqrt{2}}\left(\begin{array}{c} 
   1 \\ 
    0 \\
     0  \\
    1  
\end{array}\right)=\frac{1}{\sqrt{2}}(\ket{0}_{A}\ket{0}_{B}+\ket{1}_{A}\ket{1}_{B})
    \end{aligned}
\end{equation}


we can describe the system as being in the following
ensemble of states:


    $\ket{0}_{A}\ket{0}_{B}$ with probability $1/2$
    
    $\ket{1}_{A}\ket{1}_{B}$ with probability $1/2$
    
    
because of the $\ket{\psi^{+}}_{AB}$\\

computational basis:

\begin{equation}
    \ket{00},\ket{01},\ket{10},\ket{11},
\end{equation}


Exercise 3.6.2:\\


for X operator,eigenvalue $(\pm1)$ with the eigenvector as the follow:

\begin{equation}
    v_{1}=\frac{1}{\sqrt{2}}[\ket{0}+\ket{1}]
\end{equation}

\begin{equation}
    v_{-1}=\frac{1}{\sqrt{2}}[\ket{0}-\ket{1}]
\end{equation}


for qubit state $\ket{+},\ket{-}$,the probability of the measurement in the X basis can be calculated by Born rule.

\begin{equation}
    Pr[X=1|\ket{+}]=1
\end{equation}

\begin{equation}
    Pr[X=1|\ket{+}]=0
\end{equation}

\begin{equation}
    Pr[X=-1|\ket{-}]=0
\end{equation}

\begin{equation}
    Pr[X=-1|\ket{-}]=1
\end{equation}


In other word, the probability of two qubit system,that Alice and Bob both have the same state can be calculated by Born rule

\begin{equation}
    Pr[Alice=\ket{+},Bob=\ket{+}|\ket{\xi}]=|\bra{v_{1}v_{1}}\ket{\xi}|^{2}=1/2
\end{equation}

\begin{equation}
    Pr[Alice=\ket{-},Bob=\ket{-}|\ket{\xi}]=|\bra{v_{-1}v_{-1}}\ket{\xi}|^{2}=1/2
\end{equation}

The result was satisfied as the following probability distribution

\begin{equation}
    p_{XA,XB}(x_{A},x_{B})=\frac{1}{2}\delta(x_{A},x_{B})
\end{equation}


Quantum cloning:\\

Quantum cloning is a process that takes an arbitrary, unknown quantum state and makes an exact copy without altering the original state in any way. Quantum cloning is forbidden by the laws of quantum mechanics as shown by the no cloning theorem,


\end{document}